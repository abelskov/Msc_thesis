This thesis revolves around the reversible programming language Hermes, which is being developed at DIKU as a domain-specific language for implementing symmetric cryptography algorithms on small devices.
The reversibility of Hermes makes us able to give certain guarantees for its safety, in that certain side-channel attacks such as information leakage becomes impossible.
I describe these side-channels as well as reversibility and explain the intention of securing small devices.
TODO: I develop a new ?backend? to Hermes, which will shorten the number of steps in the compilation to my target assembly ARM64. This will give more control over side-channels as we are no longer dependant on gcc and zerostack, which are being used in the current implementation of Hermes compilation.
My new implementation is a two-step process compiling Hermes to a reversible static single assignment (RSSA) intermediate language, that will serve as a backend for translation to multiple assembly architectures - in this case ARM64. This compilation from RSSA to ARM64 constitutes the second part of the process.
I implement the 128-bit symmetric block cipher Twofish, thereby showing that Hermes is useful as a domain-specific language for implementing symmetriccryptographic algorithms.
TODO: Finally I show the correctness of my Twofish implementation by applying an existing library for property based testing.
TODO: Consider testing speed of 50 encryptions vs reference solution
