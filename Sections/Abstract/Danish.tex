Dette speciale omhandler det reversible programmeringssprog Hermes, der er udviklet på DiKU som et domæne-specifikt sprog med henblik på implementatering af symmetriske krypteringsalgoritmer på små enheder.
Reversibiliteten i Hermes gør os i stand til at give nogle garantier for sikkerheden, idet visse side-kanal angreb såsom data-lækage bliver umuliggjort. Jeg beskriver disse side-kanaler samt reversibilitet og forklarer om intentionen om at sikre små enheder.
TODO: Jeg udvikler en ny ?backend? til Hermes, som skal forkorte antallet af trin i oversættelsen til ARM64 og dermed give mere kontrol.
?Backenden? oversætter først Hermes til et reversibelt statisk mellemliggende sprog med én tildeling af hver variabel, der derfra kan oversættes til diverse mål - i dette tilfælde ARM64. 
Jeg implementerer Twofish og viser derved at Hermes er brugbart som et domæne-specifikt sprog til implementering af symmetriske krypteringsalgoritmer.
TODO: Til slut viser jeg korrektheden ved Twofish implementationen idet jeg anvender et eksisterende bibliotek til property based testing. 
