Dette speciale omhandler det reversible programmeringssprog Hermes, der er udviklet på DIKU som et domæne-specifikt sprog med henblik på implementering af symmetriske krypteringsalgoritmer på små enheder.
Reversibiliteten i Hermes gør os i stand til at give nogle garantier for sikkerheden, idet visse side-kanal angreb såsom data-lækage bliver umuliggjort.
Jeg beskriver disse side-kanaler samt reversibilitet og forklarer om intentionen om at sikre små enheder.
Jeg udvikler en oversætter fra Hermes til en reversibel statisk mellemliggende repræsentation med én tildeling af hver variabel (RSSA), som kunne blive en fælles forfader til flere forskellige assembler sprogs oversættere og hjælpe med at forkorte skridtene ved oversættelse til assembler sprog.
Jeg udvikler også en oversætter fra RSSA til ARM64 assembler sprog med abstrakte register navne.
Denne nye implementering giver mere kontrol over side-kanaler idet vi ikke længere er afhængige af \emph{gcc} og \emph{zerostack} som bliver benyttet i den nuværende Hermes til C oversættelser.
Jeg implementerer også 128-bit block cipher algoritmen Twofish i Hermes, og viser derved brugbarheden af Hermes som et domæne-specifikt sprog til implementering af symmetriske krypteringsalgoritmer.
Til slut sammenligner jeg min Twofish implementering med to reference implementeringer skrevet i hhv. Python og C. 
