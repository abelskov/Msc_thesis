Dette speciale omhandler det reversible programmeringssprog Hermes, der er udviklet på DIKU som et domæne-specifikt sprog med henblik på implementatering af symmetriske krypteringsalgoritmer på små enheder.
Reversibiliteten i Hermes gør os i stand til at give nogle garantier for sikkerheden, idet visse side-kanal angreb såsom data-lækage bliver umuliggjort. Jeg beskriver disse side-kanaler samt reversibilitet og forklarer om intentionen om at sikre små enheder.
TODO: Jeg udvikler en ny ?backend? til Hermes, som skal forkorte antallet af trin i oversættelsen til mit valg af arkitektur, som er ARM64. Det vil give mere kontrol over side-kanaler idet vi ikke længere er afhængige af gcc og zerostack som bliver benyttet i den nuværende implementation af Hermes oversættelsen.
Min nye implementation er en to-trins process som først oversætter Hermes til et reversibelt statisk mellemliggende sprog med én tildeling af hver variabel (RSSA), der derfra kan oversættes til diverse arkitekturer - i dette tilfælde ARM64. Denne oversættelse fra RSSA til ARM64 udgør andet trin af oversættelsen.
Jeg implementerer 128-bit block cipheren Twofish og viser derved at Hermes er brugbart som et domæne-specifikt sprog til implementering af symmetriske krypteringsalgoritmer.
TODO: Til slut viser jeg korrektheden ved Twofish implementationen idet jeg anvender et eksisterende bibliotek til property based testing. 
TODO: Consider testing speed of 50 encryptions vs reference solution

