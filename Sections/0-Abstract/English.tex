This thesis revolves around the reversible programming language Hermes, which is being developed at DIKU as a domain-specific language for implementing symmetric cryptography algorithms on small devices.
The reversibility of Hermes makes us able to give certain guarantees for its safety, in that certain side-channel attacks such as information leakage becomes impossible.
I describe these side-channels as well as reversibility and explain the intention of securing small devices.
I develop a compiler from Hermes to a Reversible Single Static Assignment (RSSA) representation, which could serve as a common ancestor for multiple target assembly languages and helps shorten each compilation step to my target assembly: \emph{ARM64}.
This new implementation gives more control over side-channels as we are no longer dependant on gcc and zerostack, which are being used in the current implementation of Hermes compilation.
My new implementation is a two-step process compiling Hermes to RSSA intermediate language and then from RSSA to ARM64, which constitutes the second part of the process.
I also implement the 128-bit symmetric block cipher algorithm Twofish in Hermes, thereby showing the usefulness of Hermes as a domain-specific language for implementing symmetric cryptographic algorithms.
Finally I compare my Twofish implementation with the reference implementation written in Python and show that my implementation runs at ??\% speed.
