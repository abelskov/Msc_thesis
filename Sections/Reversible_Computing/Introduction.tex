% Skal bruge: Principles of a reversible programming language - Tetsuo, holger bock, robert gluck
%             Partial evaluation of the reversible language Janus - Torben M.
%             Encryption and reversible computations - dominik taborsky, Ken F.L, Michael
%             Reversible computation and reversible PL - Tetsuo
%             TODO: A reversible programming language and its invertible self-interpreter - Yokoyama, robert gluck
%             TODO: Hermes: A reversible language for writing encryption algorithms - Torben MogensenReversible computing is an interesting paradigm with regards to security. In reversible computing one writes programs in reversible programming languages such as Janus [3].

Reversible programs can be run both forwards and backwards deterministically, calculating output from input as well as input from output [4].
What this means is that all functions are bidirectional, i.e. given some resulting output of a function, we can run the inverse of that function with the output and find the initial input.
This is especially useful for algorithms, such as encryption or encoding, where an inverse process is most often needed.
If implemented without too much overhead, this will result in smaller programs, as the encoding/encryption function is bidirectional and can be used for decoding/decryption as well.

Being bidirectional imposes some constraints on the programmer: variables local to a function must be initialized to zero and reset to zero after use.
All variable updates must be lossless, therefore not allowing modulo updates such as \textbf{x = x mod y} as there is no inverse to this operation.
Bit-shift operations must also be lossless, i.e. when bits are rotated off the right end they are inserted into the vacated bit positions on the left and vice-versa. While imposing some restrictions on the programmer, reversible programming languages has a lot to offer in terms of security: they ensure no loss of information, as all variables are cleared after use. Thus the memory will not contain any data that can be used for side-channel attacks such as information leakage.
