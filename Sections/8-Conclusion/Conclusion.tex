In this thesis we have presented the reversible programming language Hermes, and implemented a compiler from Hermes to assembly language ARM64 using RSSA as an intermediate language.
We have given an introduction to the reversible programming paradigm and explored different side-channel vulnerabilities such as information leakage and timing attacks in Chapter~\ref{chapt - Reversible-computing}.
We have discussed the ability of Hermes to protect the programmer from side-channel vulnerabilities through its type system, which ensures that information-destroying statements result in a type-error.
This results in reversability as variables are guaranteed to be zero at the start and the end of a function and we have seen that the property holds in both newly added translations to RSSA and ARM64.
Similarly, with regards to time sensitive control structures, conditional updates and swaps have constant-time execution in the newly added translations. 
In Chapter~\ref{section - RIL} we present RSSA, which lets us have more control over side-channels and enables easier register allocation. We then implement it as an intermediate language for Hermes in Chapter~\ref{chapt - implementation}.
We present our experiments in Chapter~\ref{chapt - experiments}, where we use unit tests to reveal that the compiler works as expected.

We have presented the symmetric encryption algorithm Twofish and implemented a reversible version in Chapter~\ref{chapt - Twofish}.
Implementing Twofish, a feistel network with a fairly complicated key schedule, shows how well-suited Hermes is as a domain-specific language for implementing cryptographic algorithms.
Through experimentation we have observed that the reversible Twofish implementation is approximately $20$ times faster than an existing Python version and approximately $60$ times slower than a highly optimized C version.

\section{Contributions}
Work on this thesis yields the following contributions:
\begin{itemize}
  \item We have implemented a new compiler from Hermes to reversible intermediate language RSSA, that can serve as a backend for multiple target assembly languages.
  \item We have implemented a new compiler from RSSA to ARM64 with unbounded many abstract registers. 
  \item We have implemented reversible Twofish in Hermes thereby showcasing how Hermes handles Feistel networks, and further supporting the claim that Hermes is well suited for writing symmetric encryption algorithms.
  \item We have argued for correctness of the Twofish implementation in Hermes, by unit testing and we have compared the running time against the reference implementations.
  \item We added the possibility to declare array constants in Hermes.
\end{itemize}

% Property based testing af Twofish ala det Simon brugte. Kryptere med min, dekryptere med reference impl.
% Biblioteker til GFMult så vi kan komme i mål med RSA og keySched til Twofish
% Snak om constant time multiplication fra bearssl hvordan det kan implementeres til at håndtere mul og modulo. Multiplikation er faktisk konstant tid for mange ARM.
% Register allokering vil normalt lave liveness analyse nedefra og op i et program. En register allokator for et reversibelt stykke kode er lidt specielt idet vores input registre også er vores output registre. Normalt med abstrakte registernavne ville en register allokator til reversible programmer så indsætte 0 := x (finits) for hvert registers sidste read, men her skal den holde styr på om registrer er et af output registrene og i så fald ikke gøre noget.
% hvad kan man ellers gøre og hvordan kan man komme videre herfra?
% out of bounds check
\section{Future Work}
Introducing property based testing of Twofish would be beneficial as it would be a stronger argument for correctness. Unit testing can only cover a fairly limited number of cases. Further work could encrypt with the Hermes implementation and decrypt with the reference implementation, and vice-versa, to ensure that they each others inverse.

Adding a register allocator would also be valuable, as it would allow us to compile Hermes to ARM64 code that would run on a device. A register allocator typically starts its liveness analysis from the end of a function and work its way backwards. For reversible code, however, our input registers are also output registers, meaning that we cannot simply insert finits like \lstinline{0 := x} at every registers last read. A register allocator for Hermes would have to check if a register is an output register, in which case it should not finitialize the register after its last read.

A final suggestion is an implementation of Galois Field multiplication.
The algorithm is a bit cumbersome to implement in Hermes, as we could end up multiplying with zero. An implementation would require Bennett embedding of the procedure to know what was multiplied with zero, so that we can get back the original value when running in reverse.
It would be interesting to look at an asymmetric encryption algorithm such as RSA, once Galois Field multiplication has been implemented in Hermes.
