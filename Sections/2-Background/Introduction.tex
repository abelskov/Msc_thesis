% Skal bruge: Principles of a reversible programming language - Tetsuo, holger bock, robert gluck
%             Partial evaluation of the reversible language Janus - Torben M.
%             Encryption and reversible computations - dominik taborsky, Ken F.L, Michael
%             Reversible computation and reversible PL - Tetsuo
%             TODO: A reversible programming language and its invertible self-interpreter - Yokoyama, robert gluck
%             TODO: Hermes: A reversible language for writing encryption algorithms - Torben MogensenReversible computing is an interesting paradigm with regards to security. In reversible computing one writes programs in reversible programming languages such as Janus [3].
Reversible programming is a programming paradigm dating back to the early 1960s. Gordon Moore, the CEO of Intel, whose paper from 1965 describes a doubling of the number of transistors in a computers approximately every 18 months has led many people to believe that computing power would also double roughly every two years. This has been largely true until now, but researchers argue that this is ultimately going to come to an end unless we find a solution to overheating problems.

Rolf Landaur argues in \cite{Irreversibility_paper} that any irreversible operation must result in some kind of heat dissipation and that, theoretically, with reversible operations, if operation A pushes the machine from an original state into some new state, then operation A$^{-1}$ could use the already existing energy in the system to push it back to the original state.
Such a machine has not yet been invented, but reversible programming has other interesting properties.
As discussed in Section \ref{chapt - Reversible-computing}, it can serve as protection against side-channel attacks and is also an interesting field to study to better understand quantum computing.
