% Skal bruge: Principles of a reversible programming language - Tetsuo, holger bock, robert gluck
%             Partial evaluation of the reversible language Janus - Torben M.
%             Encryption and reversible computations - dominik taborsky, Ken F.L, Michael
%             Reversible computation and reversible PL - Tetsuo
%             TODO: A reversible programming language and its invertible self-interpreter - Yokoyama, robert gluck
%             TODO: Hermes: A reversible language for writing encryption algorithms - Torben MogensenReversible computing is an interesting paradigm with regards to security. In reversible computing one writes programs in reversible programming languages such as Janus [3].
From the early 1960s up until around 2012 we've seen roughly a doubling of transistors in a dense integrated circuit every two years. The limitations associated with increasing computing power seems to be overheating problems. To combat the issue of overheating, research of the programming paradigm Reversible programming started in the early 1970s.

Rolf Landaur, one of the first to talk about reversible programming, argues in\cite{Irreversibility_paper} that any irreversible operation must result in some kind of heat dissipation and that, theoretically, with reversible operations, if operation A pushes the machine from an original state into some new state, then operation A$^{-1}$ could use the existing energy in the system to push it back to the original state.
Such a machine has not yet been invented, but reversible programming has other interesting properties.
As discussed in Section~\ref{chapt - Reversible-computing}, it can serve as protection against side-channel attacks and is also an interesting field to study to better understand quantum computing.
