\section{Our target assembly language: ARM64}
ARM64 or AArch64 is a family of reduced instruction set computing (RISC) architectures developed by Arm Holdings.
It is currently the most widely used instruction set architecture with more than 100 billion ARM processors produced as of 2017\cite{ARM_sales}. 
ARM is used in many applications, such as Raspberry Pis, self driving cars, phones, tablets etc. Apple, who has been using ARM in their iPhones and iPads recently announced they will be switching from Intel's x86\_64 to ARM64 on their MacBooks from 2020 and onwards\cite{ARM_cpus_2020}.

\subsection{ARM64 vs x86\_64}
% Addressing modes such as load unsigned byte
% Segment registers (CS: code segment, DS: data segment, SS: stack segment, ES: extra segment, FS, GS)
% x86 has variable-length instructions, while ARM is always 32 bit.
% Macro instructions such as ...
% Thumb modes for ARM ...
% CITE [http://www.informit.com/articles/article.aspx?p=1620207&seqNum=3]
Because ARM has a reduced instruction set, it is generally cheaper, less power consuming and dissipates less heat than complex instruction set computing (CISC) architectures such as x86, designed by Intel in the early 80s. This makes it a really great choice for smartphones, laptops and IOT devices.
Intel's x86 has a lot of legacy aspects such as addressing modes and segment registers that are rarely used as well as macro instructions, which can contain several instructions in one and take more work to decode. As a result of this, the ARM pipelines are generally much shorter ranging from 3 to 8 stages compared to that of x86's 14 to 32 stages.  
x86 has variable-length instruction support, whereas the ARM instruction decoder always takes a 32-bit word and just needs to test a few bits to know where to dispatch the instruction. The x86 decoder needs to read the bits in sequence, find breaks between instructions, and so on.  
x86 processors are generally seen as more powerful and a better choice for projects that require complex displays such as gaming or animation, but also require a more power and a heat sink i.e. somewhere to dump their heat dissipation. A high-end Intel i7 processor can consume as much as 130W of power, whereas ARM cores typical power consumption is approximately 5W.
ARM64 is a natural choice for Hermes since our focus is on smaller devices.
