\section{Reversible Twofish}
\subsection{Reference implementation}
Bruce Schneier, who is a co-author of the algorithm, has a Github repository\cite{GIT2F} with two implementations of Twofish: A highly optimized one in C and another one in Python2.7.
We will be focusing on the Python implementation as it is more similar to the Hermes implementation in its structure as well as being easier to read.
Its around 250 lines of code, so not awfully large but not small either.
The complexity of the code is low/medium and it is overall pretty straight forward if you know how the algorithm works.

\subsection{encrypt/decrypt}
\subsubsection{Python}
\lstinputlisting[label=app:Encrypt Python,caption=encrypt/decrypt in Python performs its encryption rounds (lifts) in a for-loop., language=Python, float=htp] {"Listings/encrypt.py"}
When looking at the encrypt/decrypt definitions in Listing~\ref{app:Encrypt Python}, we see that the only difference between encrypt and decrypt is the order in which things are done.
Encrypt uses the words K[0]-K[3] for input whitening, calls F 16 times with round number $r = 0$ to $r = 15$, and then performs output whitening with the words K[4]-K[7].
Decrypt swaps the words used for input whitening and output whitening, and lets $r$ run from $r = 15$ to $r = 0$. 

The parameters for both functions are subkeys $K$ and $S$ as well as a plaintext/ciphertext.
They both call F with five arguments: input words R[0] and R[1], the round number $r$ which is used to select which subkeys to use, and the subkeys $K$ and $S$.

\subsubsection{Hermes}
The implementation in ~\ref{app:Encrypt Hermes} is structurally similar to the Python version: input whitening followed by $16$ rounds of encryption followed by output whitening. It has the same logic (calling $F$ and rotating/XOR'ing) as the Python implementation. It uses Bennett embedding of passing along placeholder variables \lstinline{Z} and \lstinline{W}, that can store intermediate values, which we then use to reset the inplace-updated variables later with uncall.
\lstinputlisting[label=app:Encrypt Hermes,caption=encrypt in Hermes, language=Hermes, float=tp] {"Listings/encrypt.hms"}

\subsection{Galois field multiplication}
Within finite field theory, a finite field with $p^n$ elements, where $p$ is a prime, is called a Galois field.
Twofish uses a Galois field $2^8$ since it handles bytes.
Finite field multiplication occurs in elliptic curves and can be found in many other cryptographic applications such as RSA and AES.
The following examples show how Galois field multiplication is handled in Python and Hermes.

\subsubsection{Python}
We see in Listing~\ref{app:GaloisField Python} that multiplication is handled by XOR'ing variable $t$ with $b$ if a has its least significant bit set and then bit shifting $a$ and $b$.
After multiplication, we perform modulo. The result ends up being in GF($2^8$).
\lstinputlisting[label=app:GaloisField Python,caption=Galois field multiplication in Python, language=Python, float=tp] {"Listings/gfmult.py"}

\subsubsection{Hermes}
In Listing~\ref{app:GaloisField Hermes} we see an attempted implementation in Hermes.
The problem is if $b$ is equal to zero, $tmp$ will also remain zero when XOR'ed with $b$ on line 7.
That means the comparison on line 8 fails, so $a$ is not decremented.
If $a$ is not decremented, $tmp$ is not zero at the end of the function.
Another thing to note is this multiplication happens inplace on the variables $a$ and $b$, as opposed to the implementation written in Python if $b$ is equal to zero we do not save the value of $a$ for when we want to run in reverse.
This could be fixed by using Bennett embedding.
\lstinputlisting[label=app:GaloisField Hermes,caption=Galois field multiplication in Hermes, language=Hermes, float=tp] {"Listings/gfmult.hms"}
